\documentclass{beamer}

%This is an exercise in getting familiar with Beamer
%For practice, the talk 'Go for Pythonists' will be (partly) ported
%to Beamer, as a demonstration

% Copyright 2004 by Till Tantau <tantau@users.sourceforge.net>.
%
% In principle, this file can be redistributed and/or modified under
% the terms of the GNU Public License, version 2.
%
% However, this file is supposed to be a template to be modified
% for your own needs. For this reason, if you use this file as a
% template and not specifically distribute it as part of a another
% package/program, I grant the extra permission to freely copy and
% modify this file as you see fit and even to delete this copyright
% notice. 


\mode<presentation>
{
  \usetheme{boxes}
  %\setbeamercovered{transparent}
  \setbeamercovered{invisible} %default
}


\usepackage[english]{babel}
% or whatever

\usepackage[latin1]{inputenc}
% or whatever

\usepackage{times}
\usepackage[T1]{fontenc}
% Or whatever. Note that the encoding and the font should match. If T1
% does not look nice, try deleting the line with the fontenc.


\title[Go for Pythonists] % (optional, use only with long paper titles)
{Go for Pythonists}

\subtitle
{A basic introduction} % (optional)

\author[Aditya Mukerjee] % (optional, use only with lots of authors)
{Aditya Mukerjee}

\date[March 7, 2013] % (optional)
{New York Python Meetup}


% If you have a file called "university-logo-filename.xxx", where xxx
% is a graphic format that can be processed by latex or pdflatex,
% resp., then you can add a logo as follows:

 \pgfdeclareimage[height=0.5cm]{university-logo}{88x31}
\logo{\pgfuseimage{university-logo}}



% Delete this, if you do not want the table of contents to pop up at
% the beginning of each subsection:
\AtBeginSubsection[]
{
  \begin{frame}<beamer>{Overview}
    \tableofcontents[currentsection,currentsubsection]
  \end{frame}
}


% If you wish to uncover everything in a step-wise fashion, uncomment
% the following command: 

%\beamerdefaultoverlayspecification{<+->}


\begin{document}

\begin{frame}
  \titlepage
\end{frame}

\begin{frame}{Outline}
  \tableofcontents
  % You might wish to add the option [pausesections]
\end{frame}


% Since this a solution template for a generic talk, very little can
% be said about how it should be structured. However, the talk length
% of between 15min and 45min and the theme suggest that you stick to
% the following rules:  

% - Exactly two or three sections (other than the summary).
% - At *most* three subsections per section.
% - Talk about 30s to 2min per frame. So there should be between about
%   15 and 30 frames, all told.

\section{Introduction}

\subsection[Short First Subsection Name]{First Subsection Name}

\begin{frame}{}
    \begin{center}
        Python is awesome!
    \end{center}
\end{frame}

\begin{frame}{Python is awesome!}
  % - A title should summarize the slide in an understandable fashion
  %   for anyone how does not follow everything on the slide itself.

  \begin{itemize}
          \pause
  \item
      Easy to read, even for non-Pythonists
      \pause
  \item
      Only one (obvious) way to do it
      \pause
  \item
      etc.
  \end{itemize}
\end{frame}

\begin{frame}{}
    \begin{center}
        Python can be not-so-awesome\dots
    \end{center}
\end{frame}

\begin{frame}{Python can be not-so-awesome\dots}
    \begin{itemize}
            \pause
        \item
            Dynamic + strong typing = careless, fatal errors
            \pause
        \item
            Slow (compared to C)
            \pause
        \item
            Migrating code between versions requires work (ie, py3k)
            \pause
        \item
            Threading
            
    \end{itemize}

\end{frame}




\begin{frame}{Make Titles Informative.}

  You can create overlays\dots
  \begin{itemize}
  \item using the \texttt{pause} command:
    \begin{itemize}
    \item
      First item.
      \pause
    \item    
      Second item.
    \end{itemize}
  \item
    using overlay specifications:
    \begin{itemize}
    \item<3->
      First item.
    \item<4->
      Second item.
    \end{itemize}
  \item
    using the general \texttt{uncover} command:
    \begin{itemize}
      \uncover<5->{\item
        First item.}
      \uncover<6->{\item
        Second item.}
    \end{itemize}
  \end{itemize}
\end{frame}


\subsection{Second Subsection}

\begin{frame}{Make Titles Informative.}
\end{frame}

\begin{frame}{Make Titles Informative.}
\end{frame}



\section*{Summary}

\begin{frame}{Summary}

  % Keep the summary *very short*.
  \begin{itemize}
  \item
    Python has \alert{a lot of awesome things}.
  \item
    Go \alert{combines these awesome things} with an increased focus on concurrent programming.
  \item
    Go is still \alert{under development}.
  \end{itemize}
  
  % The following outlook is optional.
  \vskip0pt plus.5fill
  \begin{itemize}
  \item
      Further steps
    \begin{itemize}
    \item
        Read the Go language spec (it's readable!)
    \item
        Hop on \#go-nuts.
    \end{itemize}
  \end{itemize}
\end{frame}


\end{document}


